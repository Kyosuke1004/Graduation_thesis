\chapter{結論}
本研究ではCNNの計算リソースの削減としてBinaryConnectを用いたコーヒー生豆の良否判定を行った.また,更なる計算リソースの削減としてバイアスの2値化,および消去したネットワークを作成し検証を行った.
検証を行った4つのネットワークについて最終10Epochの平均正答率は以下の結果になった.
\begin{itemize}
  \item CNN:83.57\%
  \item BinaryConnect:82.53\%
  \item BinaryConnect(Binarize Bias):82.30\%
  \item BinaryConnect(None Bias):82.50\%
\end{itemize}

BinaryConnectではCNNにくらべ正答率のばらつきが確認できたが,平均的な精度は1ポイントほどの差にとどまっておりCNNの代替となり得ることがわかった.
また,重みのメモリ使用量だけで考えると,約4.7MBから約117KBまで削減することができるため,メモリ使用量の削減としても有効であるだろう.
バイアスを2値化したり消去した場合でも正答率は同程度の結果を示しているため,積極的に採用していきたい.

課題として,重みが1から-1,-1から1へと急激な変化をすることによる判定精度のばらつきがある.これは重みや勾配のクリッピングを行ったり,重みの3値化を行うことで改善を目指したい.また,正常豆を欠点豆だと誤判定するケースが多いため全体的な精度の向上が必要となる.

今後の展望として,3値化などの手法を用いて正答率を向上させ,FPGAやマイコンなどの計算リソースの少ないデバイスでの判別を目指す.