
% ====================================
% chapter 1
% ====================================

\chapter{背景・目的}

% ページ番号をリセット
\setcounter{page}{1}
% ページ番号の付け方をromanからarabicに変えておく
\pagenumbering{arabic}


\section{はじめに}
現在,コーヒーは世界中で嗜好品として飲まれており,その中でもスペシャルティコーヒーと呼ばれる品質を重視したコーヒーがブームを迎えている.この辺にちゃんと表入れる.そのため,質の高いコーヒー豆が求められているが,コーヒー豆において虫食いや欠けのある欠点豆と呼ばれるものが存在する.欠点豆はコーヒーの味や香りを劣化させてしまうため取り除く必要があるが,これは現在人間が手動で行なっており,ハンドピックと呼ばれている.

ハンドピックにはいくつかの問題が存在する.1つ目は生産地でハンドピックをするとコーヒー豆の価格が高騰してしまうこと.2つ目は生産地でハンドピックをしたとしても輸送や保管の段階でコーヒー豆が劣化してしまい輸入後のハンドピックは避けられないこと.3つ目はハンドピックそのものの労力が大きいことである.生産地から販売店までの商通で安価に精度良く選別することはできないため,販売店がハンドピックを行う必要があり,大きな時間と労力が費やされている.

そこで,本研究では機械学習を用いてコーヒー生豆の良否判定を行う.先行研究ではConvolutional Neural Network(以下,CNN)を用いた判別を実現している.しかし,CNNの重みは浮動小数のため計算コストが高い.将来的にFPGAやマイコンなどの計算リソースの少ないデバイスで制御するためにBinaryConnectを用いて重みの2値化を目指す.浮動小数で表されるCNNの重みは32bitや64bit分のメモリを消費するが,BinaryConnectを用いれば重みは1bitで表すことができる.これにより,メモリの消費を減らすことができるうえ,乗算機の数も減らすことができる\cite{binaryconnect}.本稿ではCNNとBinalyConnectを用いてコーヒー生豆の良否判定精度を実験により比較し,BinaryConnectの有効性を検証する.
