
% ====================================
% chapter 2
% ====================================

\chapter{BinaryConnectによる良否判定}

\section{Neural Network}
Neural Networkとはニューロンと呼ばれる人間の脳細胞をもとに作られた数理モデルである.入力に対し重みをかけた時の出力から入力の特徴を抽出し分類問題を解く機械学習の1種である.

\subsection{Neural Networkの構造}
ニューラルネットワークは入力層,出力層,隠れ層,重み,バイアスの要素からなる.
これらは図\ref{fig_NN}のような構造をしており,入力$x_i$に対し重み$w$を乗算したものにバイアス$b$を足すことで出力$y$を得るため,以下のような式で表される.
\begin{align*}
y &= w_{1}x_{1} + w_{2}x_{2} + w_{3}x_{3} + b
\end{align*}
\begin{figure}[]
  \begin{center}
    \includegraphics[scale = 0.5]{./chapter2/Neural_Network.pdf}
    \caption{Neural Networkの構造}
    \label{fig_NN}
  \end{center}
\end{figure}
ここで出力$y$を複数個にすると,図?のような構造になり,入力$x$のインデックスを$i$,出力$y$のインデックスを$j$とすると,それぞれの重みは$w_{ji}$,バイアスは$b_j$となるため計算式は以下のようになる.
\begin{align*}
y_{1} &= w_{11}x_{1} + w_{12}x_{2} + w_{13}x_{3} + b_1\\
y_{2} &= w_{21}x_{1} + w_{22}x_{2} + w_{23}x_{3} + b_2\\
y_{3} &= w_{31}x_{1} + w_{32}x_{2} + w_{33}x_{3} + b_3
\end{align*}
構造は学習モデルによって異なるため,必ずこのような形になるわけではない.そこで,入力の個数をI個とし,次の層のユニットへの出力を一般化すると以下のような式となる.
\begin{align*}
y_j = \sum^{I}_{i = 1} w_{ji}x_i + b_j
\end{align*}
更にこの出力を次の層への入力として扱うことで層を多数化し,入力層から出力層までに多数の隠れ層を追加したものをDeep Neural Networkと呼ぶ.実装する際は各層にReluやSoft Max関数などの活性化関数を追加し学習を行っている.

\subsection{学習の概要}

\section{CNN}

\section{BinaryConnect}

